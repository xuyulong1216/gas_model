\documentclass[]{ctexart}
\title{API说明}
\author{胥宇龙}
\date{\today}
\usepackage{graphicx}
\usepackage{amsmath}
\usepackage{xcolor}
\usepackage{physics}
\usepackage{listingsutf8}

\lstset{
%backgroundcolor=\color{red!50!green!50!blue!50},%代码块背景色为浅灰色
rulesepcolor= \color{gray}, %代码块边框颜色
breaklines=true,  %代码过长则换行
numbers=left, %行号在左侧显示
numberstyle= \small,%行号字体
keywordstyle= \color{blue},%关键字颜色
commentstyle=\color{gray}, %注释颜色
frame=shadowbox    %用方框框住代码块
frame=singleu
escapeinside=``  % 代码包含中文,把想插入的代码中的中文部分用 ` ` 括起来
}
\begin{document}
\subsection{创建}
使用POST方法向http://server:port/gateway/create 发送请求,请求要求为json,且具有如下格式:\{ 'port':'COMX','baud':xxxx \},将创建数据网关对象以进行后续读写操作,当创建成功时返回 201 Created ,若传入数据不合法将返回状态400 Bad Request,创建失败将返回500 Internal Server Error
\section{读取}

使用GET方法访问。
\subsection{gateway}
向http://server:port/gateway 发送请求时,将返回状态200 OK并返回一json文本,其格式为:\{'sensor0':data0,'sensor1':data1 \dots \}

向http://server:port/gateway/0xXX 等特定传感器地址发送请求时,将返回状态200 OK并返回一json文本,其格式为\{'sensorX':dataX \}

当GET请求不符合上述条件时将返回400 Bad Request 错误,在未通过http://server:port/gateway/create 创建网关对象时将返回404 Not Found

\section{预测}
向http://server:port/predictor/GAS\_TYPE/XX.xx\_YY.yy\_ZZ.zz\\\_TT.tt 发送GET请求,若与此不符则返回400 Bad Reuest,
其中GAS\_TYPE为以下三者之一: , XX.xx\_YY.yy\_ZZ.zz\_TT.tt为空间坐标与预测时长,按照此格式与顺序写入,可包含小数点
当符合时将返回200 OK并返回一json文本,格式为\{'gas\_type':GAS\_TYPE,'position': 'XX.xx\_YY.yy\_ZZ.zz',\\'time':'TT\_tt','predict\_value':value \}

\end{document}